% Russian language
\documentclass{article}
\usepackage[utf8]{inputenc}
\usepackage[russian]{babel}

% Math symbols
\usepackage{amssymb}
\usepackage{amsmath}

% Images inplace
\usepackage{float}
\usepackage{graphicx}
% Code blocks
\DeclareFixedFont{\ttb}{T1}{txtt}{bx}{n}{12} % for bold
\DeclareFixedFont{\ttm}{T1}{txtt}{m}{n}{12}  % for normal
\usepackage{color}
\definecolor{deepblue}{rgb}{0,0,0.5}
\definecolor{deepred}{rgb}{0.6,0,0}
\definecolor{deepgreen}{rgb}{0,0.5,0}
\usepackage{listings}
\newcommand\pythonstyle{\lstset{
language=Haskell,
basicstyle=\ttm,
morekeywords={self},              % Add keywords here
keywordstyle=\ttb\color{deepblue},
emph={MyClass,__init__},          % Custom highlighting
emphstyle=\ttb\color{deepred},    % Custom highlighting style
stringstyle=\color{deepgreen},
frame=tb,                         % Any extra options here
showstringspaces=false
}}
\lstnewenvironment{python}[1][]
{
\pythonstyle
\lstset{#1}
}
{}
\newcommand\pythonexternal[2][]{{
\pythonstyle
\lstinputlisting[#1]{#2}}}
\newcommand\pythoninline[1]{{\pythonstyle\lstinline!#1!}}
% Code blocks

\makeatletter
\renewcommand*\env@matrix[1][*\c@MaxMatrixCols c]{%
  \hskip -\arraycolsep
  \let\@ifnextchar\new@ifnextchar
  \array{#1}}
\makeatother

\title{Лабораторная работа №5}
\author{Ларин Егор. 4 группа 2 курс}

\begin{document}
\maketitle
\section*{Теория}
\[f(x) = \frac{e^x}{1+x^2}\]
\[a = 1, b =2, k = 4\]
\[ \boxed{\varepsilon = 10 ^{-7}}\]
\[h = \frac{b-a}{n}\]
\[I = \int_a^b f(x)dx \hspace{0.1cm} - \hspace{0.1cm} ? \]
\subsection*{Правило Рунге}
\begin{enumerate}
\item \begin{equation*}
    \begin{split}
        Q_1 &= Q(n) \\
        Q_2 &= Q(2n)
    \end{split}
\end{equation*}
\item \[R = \frac{Q_2 - Q_1} {2^m -1},\]
где $m$ -- порядок точности КФ, $n$ -- число разбиений отрезка $\left[a, b\right]$.
    \item Если $\left| R\right| \leq \varepsilon $, то прекращаем вычисления и $I \approx \hat{Q} = Q_2 + R$.
    \item В ином случае $Q_1 = Q_2, Q_2 = Q(4n)$ и возвращаемся к шагу 2.
\end{enumerate} 
\subsubsection*{КФ Симпсона}
\[
I \approx Q^C = \frac{h}{3} \cdot \sum_{i=1,2}^{n-1} f(x_{i-1}) + 4 f(x_i) + f(x_{i+1})
\]
\[m=4\]
\subsubsection*{КФ трапеций}
\[
I \approx Q^T = h \cdot \left(\frac{f(a) + f(b)}{2} + \sum_{i=1}^{n-1} f(x_i)  \right)
\]
\[m=2\]
\subsection*{КФ НАСТ}
$p(x) = 1$, значит в качестве ортогонального многочлена степени $k$ 
можно выбрать многочлен Лежандра $P_n(x)$ степени $k$.
\[P_3(x) = \frac{1}{2}\cdot \left(5x^3 - 3x\right)  \]
\[P_4(x) = \frac{1}{8}\cdot \left(35x^4 - 30x^2 + 3\right)  \]
Коэффициенты КФ могут быть найдены как
\[A_i = \frac{2}{k^2} \cdot \frac{1-x_i ^ 2}{P_3(x_i) ^ 2}, P_4(x_i) = 0, i = \overline{1, 4}\]
\begin{equation*}
    \begin{split}
        x_1 &= -0.861136311\\
        x_2 &= 0.861136311\\
        x_3 &= -0.339981043 \\
        x_4 &= 0.339981043
    \end{split}
\end{equation*}
В итоге имеем
\begin{equation*}
   \begin{split}
Q^H &=  0.34785485202283456 f(\frac{a + b}{2} + x_1  \frac{b - a}{2}) \\
&+ 0.34785485202283456 f(\frac{a + b}{2} + x_2 \frac{b - a}{2})  \\
&+ 0.6521451563287455 f(\frac{a + b}{2} + x_3  \frac{b - a}{2}) \\
&+ 0.6521451563287455 f(\frac{a + b}{2} + x_4  \frac{b - a}{2})
   \end{split} 
\end{equation*}





\section*{Листинг кода}
\begin{python}
eps = 1.0e-7
a = 1
b = 2
f x = exp x / (1 + x ^ 2)
integrateTrapezoid n = h * sum ((f a + f b) / 2 : [f x | x <- [a + h, a + 2 * h .. b - h]]) where h = (b - a) / n
integrateSimpsons n = h / 3 * sum [f (x - h) + 4 * f x + f (x + h) | x <- [a + h, a + 3 * h .. b - h]] where h = (b - a) / n
runge i m n = if abs r < eps then q2 + r else runge i m (n * 2)
  where q1 = i n
        q2 = i (2 * n)
        r = (q2 - q1) / (2 ** m -1)
xs = [-0.861136311, 0.861136311, -0.339981043, 0.339981043]
p x = 2.5 * x ** 3 - 1.5 * x
c k = (1 - (xs !! k ** 2)) / (8 * (p (xs !! k) ** 2))
sub x = (a + b) / 2 + x * (b - a) / 2
gauss = (b - a) / 2 * sum [c i * f (sub (xs !! i)) | i <- [0 .. 3]]

main = do
  print gauss
  print $ runge integrateSimpsons  4 1
  print $ runge integrateTrapezoid 2 1
\end{python}
\section*{Результаты вычислительного эксперимента}
\[Q^{H} = 1.392468989526053\]
\[Q^{C} = 1.392468998110868\]
\[Q^{T} = 1.3924689981257685\]

\begin{tabular}[H]{|c|c|c|c|c|}
  \hline
  КФ & $n$ & $h$ & $Q(n)$ & $R(n)$ \\
  \hline
 Симпсона & 1  & 1.0    & 3.09297982856427   & - \\ 
 & 2  & 0.5    & 1.392146190610417  & -0.11338890919692353 \\
 & 4  & 0.25   & 1.3924437660220899 & 1.9838360778197856e-5\\
 & 8  & 0.125  & 1.3924673509960468 & 1.5723315971290693e-6\\
 & 16 & 0.0625 & 1.3924688942332293 & 1.0288247883257402e-7\\
 & 32 & 0.03125 &1.3924689916185156 & 6.4923524186374e-9\\
  \hline
Трапеций & 1  & 1.0        & 1.4184760670078265    &         - \\
 & 2  & 0.5        & 1.3987286597097692    &          -1.3164938198704827e-3 \\
 & 4  & 0.25       & 1.39401498944401      &          -3.142446843839538e-4\\
 & 8   & 0.125     & 1.3928542606080376    &          -7.738192239815274e-5\\
 & 16  & 6.25e-2   & 1.3925652358269314    &          -1.9268318740417455e-5\\
 & 32  & 3.125e-2  & 1.39249305267062      &          -4.812210420753379e-6\\
 & 64  & 1.5625e-2 & 1.3924750114568611    &          -1.2027475839282431e-6\\
 & 128 & 7.8125e-3 & 1.3924705014395398    &          -3.0066782142531185e-7\\
 & 256 & 3.90625-4 & 1.3924693739530933 & -7.516576309522331e-8\\
  \hline


\end{tabular} 
\section*{Выводы}
Практический метод оценки численного интегрирования квадратурными формулами Рунге
действительно позволяет найти приближенное значение интеграла с заданной точности,
а квадратурная формула НАСТ построенная по методу Гаусса с весовой функцией равной единице
позволяет найти приближенное значение интеграла с соизмеримой точностью
без итерационного процесса.
\end{document}