% Russian language
\documentclass{article}
\usepackage[utf8]{inputenc}
\usepackage[russian]{babel}

% Math symbols
\usepackage{amssymb}
\usepackage{amsmath}
\usepackage{accents}

% Images inplace
\usepackage{float}
\usepackage{graphicx}
% Code blocks
\DeclareFixedFont{\ttb}{T1}{txtt}{bx}{n}{8} % for bold
\DeclareFixedFont{\ttm}{T1}{txtt}{m}{n}{8}  % for normal
\usepackage{color}
\definecolor{deepblue}{rgb}{0,0,0.5}
\definecolor{deepred}{rgb}{0.6,0,0}
\definecolor{deepgreen}{rgb}{0,0.5,0}
\usepackage{listings}
\newcommand\pythonstyle{\lstset{
language=Python,
basicstyle=\ttm,
morekeywords={self},              % Add keywords here
keywordstyle=\ttb\color{deepblue},
emph={MyClass,__init__},          % Custom highlighting
emphstyle=\ttb\color{deepred},    % Custom highlighting style
stringstyle=\color{deepgreen},
frame=tb,                         % Any extra options here
showstringspaces=false
}}
\lstnewenvironment{python}[1][]
{
\pythonstyle
\lstset{#1}
}
{}
\newcommand\pythonexternal[2][]{{
\pythonstyle
\lstinputlisting[#1]{#2}}}
\newcommand\pythoninline[1]{{\pythonstyle\lstinline!#1!}}
% Code blocks

\makeatletter
\renewcommand*\env@matrix[1][*\c@MaxMatrixCols c]{%
  \hskip -\arraycolsep
  \let\@ifnextchar\new@ifnextchar
  \array{#1}}
\makeatother

\title{Лабораторная работа №4}
\author{Ларин Егор. 4 группа 3 курс}

\begin{document}
\maketitle
\section*{Теория}
\subsection*{Задача}
\begin{equation*}
    \frac{\partial u}{\partial t} = \frac{\partial ^2u}{\partial x^2} +t^2 \sin xt + x \cos xt, 0 < x <1, 0< t\leq0.5,
\end{equation*}

$$\begin{cases}
    u(x,0) = 0, 0\leq x \leq 1,\\
    u(0,t) = 0, 0\leq t \leq \frac{1}{2},\\
    u(1,t) = \sin(t), 0\leq t \leq \frac{1}{2}.
\end{cases}$$

\subsection*{Точное решение}
\begin{equation}
    u(x,t)  = \sin(x t)
\end{equation}

\subsection*{Сетка}
\begin{itemize}
    \item $\tau$ -- шаг сетки по времени, $N_2 = \frac{1}{2\tau}$.
    \item $h$ -- шаг сетки по пространству, $N_1 = \frac{1}{h}$.
    \item $\varphi_i^j = f(x_i, t_{j+\frac{1}{2}})$
\end{itemize}

\subsection*{Граничные условия}
$$\begin{cases}
    y_{i}^0 = 0, i = \overline{0, N_1}\\
    y_{0}^j = 0, j = \overline{0, N_2}\\
    y_{N_1}^j = \sin (j\tau), j = \overline{0, N_2}\\
\end{cases}$$

\subsection*{Разностная схема}
\begin{enumerate}
    \item Явная $\frac{y_i^{j+1}-y_i^j}{\tau} = \Lambda y_i^j + \varphi_i^j$
    \item Чисто неявная $\frac{y_i^{j+1}-y_i^j}{\tau} =  \Lambda y_i^{j+1}+\varphi_i^j$
    \item Кранка-Николсона $\frac{y_i^{j+1}-y_i^j}{\tau} = \frac{1}{2}(\Lambda y_i^j + \Lambda y_i^{j+1})+\varphi_i^j$
\end{enumerate}

\subsection*{Устойчивость и порядок аппроксимации}
\begin{itemize}
    \item Явная -- $O(\tau + h^2)$, не является абсолютно устойчивой.
    \item Чисто -- неявная $O(\tau + h^2)$, является абсолютно устойчивой.
    \item Кранка-Николсона -- $O(\tau^2 + h^2)$, является абсолютно устойчивой.
\end{itemize}

\section*{Листинг кода}
\begin{python}
import numpy as np
import matplotlib.pyplot as plt
from mpl_toolkits.mplot3d import Axes3D

a = 0.
b = 1.
c = 0.
d = 0.5

draw_plots = False

def u(x, t):
    return np.sin(t * x)

def u0(x):
    return 0

def u1(t):
    return 0.

def u2(t):
    return np.sin(t)
    
def f(x, t):
    return t**2 * np.sin(x*t) + x * np.cos(x * t)

def TDMA(a, b, c, f):
    beta = []
    l = len(c) - 1
    alfa = [b[0] / c[0]]
    beta.append(f[0] / c[0])
    for i in range(1, l):
        check = (c[i] - a[i - 1] * alfa[i - 1])
        if abs(check) < 1e-19: exit()
        alfa.append(b[i] / check)
        beta.append((f[i] + a[i - 1] * beta[i - 1]) / check)
    beta.append((f[l] + a[l - 1] * beta[l - 1]) / (c[l] - a[l -1] * alfa[l - 1]))
    y = [0] * (l + 1)
    y[l] = beta[l]
    for i in range(l - 1, -1, -1):
        y[i] = alfa[i] * y[i + 1] + beta[i]
    return np.array(y)

def plot(y):
    data = np.array(y)
    length = data.shape[0]
    width = data.shape[1]
    x, y = np.meshgrid(np.arange(width), np.arange(length))
    ax = plt.axes(projection='3d')
    ax.plot_surface(x, y, data)
    if draw_plots:
        plt.show()

def build_exact(h, tau):
    n = int((b - a) / h) + 1
    m = int((d - c) / tau) + 1
    ys = np.zeros((n, m))
    for i in range(n):
        for j in range(m):
            ys[i, j] = u(h*i, tau*j)
    return ys

def build_explicit(h, tau):
    xs = np.linspace(a, b, int((b - a) / h) + 1)
    ts = np.linspace(c, d, int((d - c) / tau) + 1)
    ys = np.zeros((len(xs), len(ts)))
    
    for i in range(len(xs)):
        ys[i, 0] = u0(xs[i])
        
    for j in range(len(ts) - 1):
        ys[0, j + 1] = u1(ts[j + 1])
        
        for i in range(1, len(xs) - 1 ):
            a_i = (ys[i + 1, j] - 2 * ys[i, j] + ys[i - 1, j]) / (h * h)
            b_i = f(xs[i], ts[j])
            ys[i, j + 1] = tau * (a_i + b_i) + ys[i, j]
            
        ys[len(xs) - 1, j + 1] = u2(ts[j + 1])
    
    plot(ys)

    return ys

def build_implicit(h, tau):
    xs = np.linspace(a, b, int((b - a) / h) + 1)
    ts = np.linspace(c, d, int((d - c) / tau) + 1)
    ys = np.zeros((len(xs), len(ts)))
    
    for i in range(len(xs)):
        ys[i, 0] = u0(xs[i])
        
    for j in range(len(ts)-1):
        c_i = [1]
        c_i = c_i + [1 / tau + 2 / (h * h)] * (len(xs) - 2)
        c_i.append(1)
        
        a_i = []
        a_i = a_i + [-1 / (h * h)] * (len(xs) - 2)
        a_i.append(0)
        
        b_i = [0]
        b_i = b_i + [-1 / (h * h)] * (len(xs) - 2)
        
        f_i = [u1(ts[j + 1])]
        for i in range(1, len(xs) - 1):
            l = f(xs[i], ts[j + 1])
            f_i.append((ys[i, j] / tau) + l)
        f_i.append(u2(ts[j + 1]))
        ys[:, j + 1] = TDMA(-np.array(a_i), -np.array(b_i), c_i, f_i)

    plot(ys)
    return ys


def build_nickolson(h, tau):
    xs = np.linspace(a, b, int((b - a) / h) + 1)
    ts = np.linspace(c, d, int((d - c) / tau) + 1)
    ys = np.zeros((len(xs), len(ts)))
    
    for i in range(len(xs)):
        ys[i, 0] = u0(xs[i])
        
    for j in range(len(ts)-1):
        c_i = [1]
        c_i = c_i + [1 / tau + 1 / (h * h)] * (len(xs) - 2)
        c_i.append(1)
        
        a_i = []
        a_i = a_i + [-1 / (2 * h * h)] * (len(xs) - 2)
        a_i.append(0)
        
        b_i = [0]
        b_i = b_i + [-1 / (2 * h * h)] * (len(xs) - 2)
        
        f_i = [u1(ts[j + 1])]
        for i in range(1, len(xs) - 1):
            l = f(xs[i], ts[j] + 0.5 * tau)
            v = (1 / (2 * h * h)) * (ys[i + 1, j] - 2 * ys[i, j] + ys[i - 1, j])
            f_i.append((ys[i, j] / tau) + l + v)
        f_i.append(u2(ts[j + 1]))
        ys[:, j + 1] = TDMA(-np.array(a_i), -np.array(b_i), c_i, f_i) 

    plot(ys)
    return ys

print(f'Explicit: {np.max(np.abs(build_exact(0.1,0.1) -build_explicit(0.1,0.1)))}')
print(f'Explicit: {np.max(np.abs(build_exact(0.1,0.005) - build_explicit(0.1,0.005)))}')
print(f"Implicit: {np.max(np.abs(build_exact(0.1,0.1) - build_implicit(0.1,0.1)))}")
print(f"Nickolson: {np.max(np.abs(build_exact(0.1,0.1) - build_nickolson(0.1, 0.1)))}")
\end{python}

\section*{Графики}
\begin{figure}[H]
\includegraphics[width=0.9\textwidth]{unstable.png}
\caption{Явная неустойчивая}
\includegraphics[width=0.9\textwidth]{stable.png}
\caption{Явная устойчивая}
\end{figure}
\begin{figure}[H]
\includegraphics[width=0.9\textwidth]{implicit.png}
\caption{Чисто неявная}
\includegraphics[width=0.9\textwidth]{nick.png}
\caption{Кранка-Николсона}
\end{figure}
\begin{figure}[H]
\includegraphics[width=0.9\textwidth]{exact.png}
\caption{Точное решение}
\end{figure}

\section*{Результаты вычислительного эксперимента}
\begin{tabular}[H]{|l|l|}
  \hline
  Схема & $\max \left|y_i^j - u(x_i, t_j)\right| $ \\
  \hline
  Явная (нейстойчивая) &  18.08729087566136 \\
  \hline
  Явная (устойчивая) &   2.770470845148143e-05\\
  \hline
  Чисто неявная& 0.0004969953632136814\\
  \hline
  Кранка-Николсона & 0.00017371480558825425 \\
  \hline
\end{tabular} 
\end{document}